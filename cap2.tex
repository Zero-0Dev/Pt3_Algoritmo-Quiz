\chapter{Fluxograma}
\section{Versão Inicial}
\begin{figure}[!h]
\caption{Fluxograma inicial do projeto}
\label{Teste - 1}
\centering
\includegraphics[scale=0.8]{Fluxo1.png}
\end{figure}
\pagebreak
\section{Versão Final}
\begin{figure}[!h]
\caption{Fluxograma final do projeto}
\label{Teste - 1}
\centering
\includegraphics[scale=0.45]{Fluxo2.png}
\end{figure}
\begin{description}
    \item[1. Início:] O processo começa no círculo "Inicio". 
    \item[2. Pergunta inicial: Gostaria de iniciar o game? ] Decisão: Se a resposta for sim, o jogo inicia. 
    Se for não, há uma pergunta sobre sair ou iniciar as configurações. 
    \item[3. Se o jogador quer iniciar o game:] O sistema mostra uma explicação sobre o jogo. 
    O timer inicia (indicando o começo da contagem do tempo para o jogo). 
    \item[4. Níveis de perguntas: ]O jogador passa por níveis numerados (Nível 1 a Nível 5). 
    Em cada nível, o jogador receberá 3 perguntas sortidas por nível para responder. 
    \item[5. Calculo de Precisão: ]O sistema calcula a precisão do jogador de acordo com seu desempenho. 
    \item[6. Resultado:] O sistema exibe o resultado com o tempo decorrido e a precisão. 
    \item[7. Se o jogador não quer iniciar o game:]Pergunta se deseja sair imediatamente ou acessar as configurações. 
    Se acessar configurações, é possível fazer operações CRUD (criar, ler, atualizar e deletar as perguntas e respostas do quiz). 
\end{description}